\documentclass{report}

\title{The LTSmin toolset}
\author{Stefan Blom, Jaco van de Pol and Michael Weber}

\usepackage{tabularx}
\usepackage[all]{xypic}

\newcommand{\struct}[1]{\ensuremath{\langle#1\rangle}}

%% Set up theorems and such...

\usepackage{theorem}
\theoremstyle{plain}
\theoremheaderfont{\bfseries\upshape}
\theorembodyfont{\mdseries\upshape}

\newtheorem{theorem}{Theorem}[section]
\newtheorem{intermezzo}[theorem]{Intermezzo}
\newtheorem{example}[theorem]{Example}
\newtheorem{lemma}[theorem]{Lemma}
\newtheorem{corollary}[theorem]{Corollary}
\newtheorem{conjecture}[theorem]{Conjecture}
\newtheorem{definition}[theorem]{Definition}
\newtheorem{proposition}[theorem]{Proposition}
\newtheorem{fact}[theorem]{Fact}
\newtheorem{remark}[theorem]{Remark}
\newenvironment{proof}[1][Proof.~]{{\bf #1}}{\mbox{}\hfill$\Box$\medskip}

%% End theorem setup.

\begin{document}

\maketitle

\tableofcontents


\chapter{Design Rationale}

\section{Management of data values.}

All data type functionality is provided by the languages modules.
The tool level itself can only manage integers, chunks and strings.
Thus, every other data type has to be serialized to a chunk
and/or pretty printed to a string.

While this approach makes the tool design simpler, it does add cost
in time and memory: (de)serialization is expensive in time and the
serialized form often takes up a lot more memory than the real data
type. For example, it takes time to print/parse ATerms. Moreover,
ATerm use maximal sharing to reduce the memory footprint which is not
possible in the serialized form.

Because of the time cost, we try to avoid (de)serialization by
keeping a global integer to value mapping, so we can send integers
instead of chunks. Because of the memory cost, duplication of
serialized values should be avoided. That is, tools that start language
modules should use the language modules to store values in native form
and the file format should store all values for each type once.

This requirement is somewhat at odds with the fact that the performance
of the symbolic data structures depends on the integers used.
A variable with integers $1,2,3,4$ performs better than one with
$1,5,45,81$, simply because the first set can be represented with less bits.
This is solved by introducing a global map for each variable that maps
the used indices to a consecutive range.


\chapter{File Formats}

\section{DIRectory format}

\section{Vector Format}

A vector format file, stores the states and transitions
in $W$ partitions each. For each partition, a list of source states and state labels is stored
as one file per state slot and state label. This defines a segmented state numbering.
The transitions are stored as one file containing the state number of a source state
plus one file per destination state slot and edge label.

\par\noindent\begin{tabularx}{\textwidth}{lX}
\verb+info+ & Stores all information that is needed but does not need a file of its own.
\\
\verb+SV-+$i$\verb+-+$k$ & Contains the State Vector slot value for segment $i$ and slots $k$.
If $N = 1$ and $S^i$ is of the form $\{ 0 , \cdots , |S^i| -1 \}$ then the
file \verb+SV-1-1+ can be omitted because it is a simple sequence $0,1,\cdots,|S^1|-1$.
\\
\verb+SL-+$i$\verb+-+{\it name} & Contains the value of state label {\it name} for the states in
segment $i$.
\\
\verb+ES-+$i$ & Contains the Edge Source state number for transitions starting in segment $i$.
\\
\verb+ED-+$i$\verb+-+$k$ & Contains the Edge Destination state slot value for 
segment $i$ and slot $k$.
\\
\verb+EL-+$i$\verb+-+{\it name} &  Contain the Edge Label value for the edge label {\it name} for
edges starting in segment $i$.
\\
\verb+CT-+{\it name} & Stores serialized values of the sort {\it name}.
\\
\verb+PP-+{\it name} & Stores pretty printed values of the sort {\it name}.
\end{tabularx}


\end{document}

