\documentclass{beamer}
%\documentclass[handout]{beamer}

\usepackage{graphics}
\usepackage{pgf,pgfarrows,pgfnodes,pgfautomata,pgfheaps,pgfshade}
%\pgfdeclareimage[height=1em]{ec-moan-logo}{ec-moan-logo}
%\logo{\pgfuseimage{ec-moan-logo}}


\date{Progress report 16-10-2008}

\usepackage[all]{xy}

%\usetheme[dutch,navbarbot]{utwente}
\usetheme[english]{utwente}

\usepackage[english]{babel}
\usepackage[latin1]{inputenc}

\usepackage{times}
\usepackage[T1]{fontenc}

%\usepackage[xy]{arrows}
\usepackage{amssymb}

\usepackage{colortbl}

\title{The Distributed Tool Chain\\Cluster Questions}
\author{Stefan Blom}

\begin{document}

\frame[plain]{\titlepage}


%\begin{frame}
%\frametitle{Outline}
%\begin{itemize}
%\item Introduction
%\end{itemize}
%\end{frame}


\begin{frame}
\frametitle{end goal}
\begin{itemize}
\item Distributed state space generation for black box and grey box models.
\item Distributed bisimulation minimization:
\begin{itemize}
\item strong bisimulation
\item branching bisimulation
\item markovian bisimulation
\item $\tau$-cycle elimination
\end{itemize}
\item On-disk state space generation ready.
\end{itemize}
\end{frame}

\begin{frame}
\frametitle{architecture}
\begin{center}
\setlength{\arrayrulewidth}{1pt}
\begin{tabular}{|@{\phantom{\Huge{g}}}l@{\phantom{\Huge{g}}}|}
\hline
(Distributed) Application
\\\hline
(Distributed) LTS library
\\\hline
(Distributed) support framework
\\\hline
\end{tabular}
\end{center}
\end{frame}

\begin{frame}
\frametitle{LTS library}
Manage a segmented LTS with:
\begin{itemize}
\item State labels including full state vectors and atomic propositions.
\item Edges labels including actions and rates.
\end{itemize}
\end{frame}

\begin{frame}
\frametitle{support framework}
The support frame work should:
\begin{itemize}
\item provide efficient IO abstraction layer
\item provide small replicated hash table (leaf DB,label DB, small node DB)
\item provide big (optionally cached) hash table (signatures, big node DB)
\item transparent compression/decompression of data
\end{itemize}
\end{frame}

\begin{frame}
\frametitle{File formats}
\begin{itemize}
\item Standardized data stream header to enable coding and/or compression
of a single data stream.
\item Support for multiple stream archives:
\begin{itemize}
\item By storing each stream as a separate file. E.g. in a directory or by using a pattern.
\item By storing the streams in one stream in such a way that it is easy to use as a pipeline.
\item By storing the streams in one file with an index to make efficient
 access to each stream possible.
\end{itemize}
\end{itemize}
\end{frame}

\begin{frame}
\frametitle{Experiments}
\begin{itemize}
\item Run experiments to assess the quality of block allocation algorithms on various
file system implementations by measuring the speed of read and write orders and number of files.
\item Run experiments to asses the effect of the cluster size on sequentialized reads.
(E.g. tweety: 4k: 8.53MB/s; 256k: 11.66MB/s; 1024k: 19.39MB/s; $\infty$: 24.53MB/s).
\item Run experiments to assess compression/decompression speeds for gzip/bzip2/lzo
to be able to determine if parallel decompression is necessary or not.
\end{itemize}
\end{frame}

\begin{frame}
\frametitle{Initial results}
Interleaved writing of 4 files of 1GB with 4k block size:
\begin{tabular}{llrrr}
\hline
&&write&par&seq
\\\hline
tweety & ext3 & 22.22 & 9.32 & 24.56
\\\hline
tweety & xfs & 23.28 & 8.58 & 24.53
\\\hline
tweety & vfat & 5.89 & 14.75 & 6.43
\\\hline
\end{tabular}
Tweety single 4GB file yields write 19.39 MB/s and read 24.18 MB/s.
\end{frame}


\begin{frame}
\frametitle{Cluster}
\begin{itemize}
\item Backup policy for home directories?
\\
Yes or no ; if yes include list or exclude list  and how: direct or rsync to EWI home dir.
\item Access to big mems.
\\
What do we require (direct connection, VPN connection, hidden behind twickel)?
\\
In other words: ssh weldam or ssh twickel ssh weldam?
\item Access from cluster:
\\
Which nodes, besides twickel, can see the internet: big, data, compute?
\item How much root access do we demand?
\\
E.g. configuration of torque/maui? File systems on data nodes?
\\
(Jan Veninga raised the security question.)
\end{itemize}
\end{frame}

\begin{frame}
\frametitle{Roadmap}
\begin{enumerate}
\item Low level IO library and tools.
\item Library and tools for managing LTS's.
\item Rewrite MPI event loop from probe to wait.
\item Distributed generator for testing purposes.
\item Linking in the new Grey Box interface.
\item Porting Instantiators to the new framework.
\item Porting the distributed reduction tools.
\item Generating abstractions for big biological ODE's.
\item Add property specific reductions to reduction tools.
\end{enumerate}
\end{frame}

\begin{frame}
\frametitle{Sequential Tools}
\begin{description}
\item[sdd] Tool that can add/remove/change the compression of a stream.
\item[mkar] Build an archive from a list of files.
\item[gsf2ar] Convert an archive stream to a (set of) file(s).
\item[ar2gsf] Convert a (set of) file(s) to an archive stream.
\item[ar2bcg] Convert a (set of) file(s) containing a single segment SVC2 to a BCG file.
\item[bcg2gsf] Convert a BCG file to a single segment SVC2 as an archive stream.
\end{description}
\end{frame}

\begin{frame}[fragile]
\frametitle{MPI Tools}
\begin{description}
\item[mpi-inst] Distributed state space generator based on \verb+libstep+.
\item[mpi\underline{~}min] Distributed strong/branching bisimulation reduction.
\end{description}
\end{frame}

\end{document}



